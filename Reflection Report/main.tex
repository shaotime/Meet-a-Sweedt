\documentclass[a4paper,11pt]{article}
\usepackage{fancyhdr}
\usepackage[swedish]{babel}
\usepackage[utf8]{inputenc}
\usepackage{graphicx}
\usepackage{float}
\usepackage{amsmath}
\usepackage[mediumspace,mediumqspace,Grey,squaren]{SIunits}
\usepackage{listings} 
\setlength{\parindent}{0pt}
\begin{document}
\begin{titlepage}

\title{\Huge{Reflection Report} \\[0.1cm] \Large{Software Engineering Project} \\ [0.9cm] \Large{ \emph{Android Application - Meet a Sweedt}} \\[0.2cm] \Large{ \emph{Scrum project}}}
\maketitle
\begin{center}
\large{ \textbf{Contributors:} } 
\newline \newline
\begin{tabular}{|l|l|l|} \hline
Name      & CID    & ID number  \\   \hline 
Ajla Cano & ajlac  & 941104    \\ \hline 
NA  & NA & NR     \\ \hline
NA  & NA & NR     \\ \hline 
NA  & NA & NR     \\ \hline 
NA  & NA & NR   \\  \hline 
NA  & NA & NR  \\  \hline
\end{tabular}
\end{center}
\thispagestyle{empty}
\end{titlepage}

\clearpage
%------------------------------
\section{Introduction}
[KORT INTRODUKTION SOM TÄCKER LITE AV ALLT SOM NÄMNS I RAPPORTEN 
\newline
Håkans note på github: 
Each part worth 3 points has an allocation strategy where 0 represents failed delivery, 1 equals major remarks, 2 signifies minor remarks and 3 no remarks. Including sprrint retrospectives for all sprints gives 1 point, including the burn-down chart is also worth 1 point.
\\ 
Reflection is here defined as “assessment of what is in relation to what might or should be and includes feedback designed to reduce the gap” (R. Smith. Formative Evaluation and the Scholarship of Teaching and Learning. New Directions for Teaching and Learning, vol. 88, 2001, pp. 51-62). This means that you should describe the situation as it is, what you would like it to be as well as a realistic way to get there.]
\section{Application of Scrum}
[Roles, team work and social contract (relates to D1B)
Used practices (pair programming, stand-up meetings, etc.)
Time distribution (person / role / tasks etc.)
Effort and velocity and task breakdown]
\section{Sprint retrospectives}
[Reflection on the sprint retrospectives \\
Documentation of sprint retrospectives, 0-1p\\
Reflection on the sprint reviews]
\section{Tools and technologies}
[Best practices for using new tools and technologies]
\section{Literature and guest lectures}
[Hur vi använde oss av literature and guest lectures i vårt projekt]

\pagenumbering{arabic}
\setcounter{page}{1}
\end{document}
